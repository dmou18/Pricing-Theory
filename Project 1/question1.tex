\documentclass[12pt, letterpaper]{article}
\usepackage[utf8]{inputenc}
\usepackage{amsthm}
\usepackage{amsmath}
\usepackage{amsfonts}
\usepackage{amssymb}
\usepackage{mathtools}

\title{STA 2503/MMF 1928 Project 1 - American Options}
\author{Zixun Zhai, Siyu Jia, Dixin Mou}
\date{2022/10/01}

\begin{document}
\begin{titlepage}
  \begin{center}
      \vspace*{7cm}

      \textbf{STA 2503/MMF 1928 Project 1 - American Options}

      % \vspace{0.5cm}
      %  Thesis Subtitle
           
      \vspace{1.5cm}

      \textbf{Zixun Zhai, Siyu Jia, Dixin Mou}

      \vfill
           
      % A thesis presented for the degree of\\
      % Doctor of Philosophy
           
      \vspace{0.8cm}
           
      % Department Name\\
      University of Toronto\\
      Toronto, Ontario, Canada\\
      October $1^{st}$, 2022 
           
  \end{center}
\end{titlepage}



\part*{Question 1}
\begin{proof}
We start the proof by considering \[X^{(N)} = \log(\frac{S_T}{S_0}) = \sum_{n=1}^N (r\Delta t+\sigma \sqrt[]{\Delta t} \epsilon_n)\]
The m.g.f of $X^{(N)}$ is: \\
% \begin{equation}
% \begin{split}
\begin{align*}
  \mathbb{E}^\mathbb{P}[e^{\mu X^{(N)}}] & =  \mathbb{E}^\mathbb{P}[e^{\sum_{n=1}^N (\mu r\Delta t+\mu\sigma \sqrt[]{\Delta t} \epsilon_n)}] \\
  & = \mathbb{E}^\mathbb{P} [\prod_{n=1}^N e^{\mu r\Delta t+\mu\sigma \sqrt[]{\Delta t} \epsilon_n}] \\
  & = (\mathbb{E}^\mathbb{P} [e^{\mu r\Delta t+\mu\sigma \sqrt[]{\Delta t} \epsilon_n}])^N \tag*{as $\epsilon_n$ is i.i.d}
\end{align*}
% \end{split} 
% \end{equation}
We then investigate the inner term $\mathbb{E}^\mathbb{P} [e^{\mu r\Delta t+\mu\sigma \sqrt[]{\Delta t} \epsilon_n}]$:
\begin{align*}
  \mathbb{E}^\mathbb{P} [e^{\mu r\Delta t+\mu\sigma \sqrt{\Delta t} \epsilon_n}] &= e^{\mu r\Delta t +\mu\sigma \sqrt{\Delta t}} \cdot \frac{1}{2} (1 + \frac{(\mu -r) - \frac{1}{2}\sigma^2}{\sigma} \sqrt{\Delta t}) \\ 
    & \quad + e^{\mu r\Delta t-\mu\sigma \sqrt[]{\Delta t}} \cdot \frac{1}{2} (1 - \frac{(\mu -r) - \frac{1}{2}\sigma^2}{\sigma} \sqrt{\Delta t}) \\
    &= [1+\mu r\Delta t+\mu \sigma \sqrt{\Delta t} +\frac{1}{2}\mu^2\sigma^2\Delta t + o(\Delta t)] \cdot \frac{1}{2} (1 + \frac{(\mu -r) - \frac{1}{2}\sigma^2}{\sigma} \sqrt{\Delta t}) \\ 
    & \quad + [1+\mu r\Delta t-\mu \sigma \sqrt{\Delta t} +\frac{1}{2}\mu^2\sigma^2\Delta t + o(\Delta t)] \cdot \frac{1}{2} (1 - \frac{(\mu -r) - \frac{1}{2}\sigma^2}{\sigma} \sqrt{\Delta t}) 
\end{align*}

Note that we collect all the terms containing $\Delta t$ with order 2 or higher to be $o(\Delta t)$ because as $N\rightarrow \infty$ 
and $\Delta t \rightarrow 0$, $o(\Delta t)$ will converge to 0 faster than $\Delta t$ or $\Delta t$ with lower orders.
\\ \\ 
We then continue the algebra manipulation:
\begin{align*}
  \mathbb{E}^\mathbb{P} [e^{\mu r\Delta t+\mu\sigma \sqrt{\Delta t} \epsilon_n}] &= [1+\mu r\Delta t+\mu \sigma \sqrt{\Delta t} +\frac{1}{2}\mu^2\sigma^2\Delta t + o(\Delta t)] \cdot \frac{1}{2} (1 + \frac{(\mu -r) - \frac{1}{2}\sigma^2}{\sigma} \sqrt{\Delta t}) \\ 
    & \quad + [1+\mu r\Delta t-\mu \sigma \sqrt{\Delta t} +\frac{1}{2}\mu^2\sigma^2\Delta t + o(\Delta t)] \cdot \frac{1}{2} (1 - \frac{(\mu -r) - \frac{1}{2}\sigma^2}{\sigma} \sqrt{\Delta t}) \\
    &= \frac{1}{2}[ 1 + \frac{(\mu - r) - \frac{1}{2}\sigma^2}{\sigma}\sqrt{\Delta t} + \mu r \Delta t + \mu \sigma \sqrt{\Delta t} \\
    & \quad + \mu((\mu -r) - \frac{1}{2}\sigma^2)\Delta t + \frac{1}{2}\mu^2 \sigma^2 \Delta t \\
    & \quad + 1 - \frac{(\mu - r) - \frac{1}{2}\sigma^2}{\sigma}\sqrt{\Delta t} + \mu r \Delta t - \mu \sigma \sqrt{\Delta t} \\
    & \quad + \mu((\mu -r) - \frac{1}{2}\sigma^2)\Delta t + \frac{1}{2}\mu^2 \sigma^2 \Delta t + o(\Delta t)] \\
    &= 1 + \mu r \Delta t + \mu((\mu - r) - \frac{1}{2}\sigma^2)\Delta t + \frac{1}{2}\mu^2 \sigma^2 \Delta t + o(\Delta t) \\
    &= 1 + (\mu r + \mu^2 - \mu r -\frac{1}{2}\mu \sigma^2 + \frac{1}{2}\mu^2 \sigma^2)\Delta t + o(\Delta t) \\
    &= 1 + (\mu (\mu - \frac{1}{2}\sigma^2) + \frac{1}{2}\mu^2 \sigma^2)\Delta t + o(\Delta t)\\
    &= e^{(\mu (\mu - \frac{1}{2}\sigma^2) + \frac{1}{2}\mu^2 \sigma^2)\Delta t} + o(\Delta t)
\end{align*} 
As a result:
\begin{align*}
  \mathbb{E}^\mathbb{P}[e^{\mu X^{(N)}}] & = (e^{(\mu (\mu - \frac{1}{2}\sigma^2) + \frac{1}{2}\mu^2 \sigma^2)\Delta t} + o(\Delta t))^N\\
    &=  e^{(\mu (\mu - \frac{1}{2}\sigma^2)T + \frac{1}{2}\mu^2 \sigma^2T)} \tag*{as $N \rightarrow \infty$ and $\Delta t = \frac{T}{N}$}
\end{align*}
We notice that the m.g.f of the $X^{(N)}$ is equal to the m.g.f of a random variable $Y$ whcih follows the normal distribution with mean to be $(\mu - \frac{1}{2}\sigma^2)T$ and
variance to be $\sigma^2 T$.
\\ \\
Thus, we prove that:
\[X^{(N)} \xrightarrow[N \rightarrow \inf]{d} (\mu - \frac{1}{2}\sigma^2)T + \sigma^2 TZ\]
where 
\[Z \stackrel{\mathbb{P}}{\sim} \mathcal{N}(0, 1)\] 
\end{proof}

\end{document}